\chapter{Considerações Finais}

O objetivo geral proposto foi cumprido com sucesso. Foi feita uma implementação de uma framework simples, nos moldes da framework Love2D e XNA, com simplificações. Alem disso foram implementadas algumas funcionalidades não propostas no objetivo que facilitam profundamente a implementação de um jogo, em especial o sistema de eventos e o de chamadas temporais.

O projeto pode ser continuado através da implementação de mais funcionalidades nele para expandir ainda mais o leque de possibilidades de implementação de jogos com a framework.

Algumas funcionalidades já planejadas para a framework são:

\subsection{Alta prioridade}
\begin{itemize}
  \item Permitir o uso de uma biblioteca de UI imediata para permitir alterações em parametros do jogo em tempo real e criação de uma interface de desenvolvimento durante o jogo.
  \item Implementação de um importador de modelos 3d e criação de mais algumas formas geométricas como: triangulo, quadrado, esfera e piramide.
  \item Implementar a utilização de texturas, para que seja possível que os objetos tenham não só uma unica cor, mas desenhos mais complexos.
\end{itemize}

\subsection{Media prioridade}
\begin{itemize}
  \item Fazer com que a camera possa renderizar para lugares diferentes da tela (fazer 2 camera que mostram coisas diferentes em cada metade da tela),
  \item Melhorar a performance para suportar ainda mais objetos em cena.
\end{itemize}

\subsection{Baixa prioridade}
\begin{itemize}
  \item Extensão da interface input\_handler para receber inputs de joysticks, e tambem criação de uma classe KeyboardMouseInput, que junta teclado e mouse em uma única funcionalidade.
  \item Expandindo as melhorias da camera e da textura, porém outra tarefa de igual dificuldade, seria poder colocar a imagem da camera em uma textura, abrindo margem para diversas possibilidades graficas, como espelhos.
\end{itemize}
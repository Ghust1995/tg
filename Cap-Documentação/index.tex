\chapter{Documentação}
\label{cap:Documentação}

Nesse capítulo as principais interfaces publicas da aplicação serão explicadas. Como explicado no \nameref{cap:desenvolvimento}, foram implementados duas bibliotecas: \nameref{doc:graphics}, \nameref{doc:time}, e 6 classes: \nameref{doc:game}, \nameref{doc:transform}, \nameref{doc:camera}, \nameref{doc:events}, \nameref{doc:scheduler}, \nameref{doc:keyboardinput}, \nameref{doc:mouseinput},

\section{Classe Game}
\label{doc:game}

A classe game é a classe principal executada pela framework, ela é composta de 3 métodos que devem ser implementados: 

\subsection{void Load()}
Chamada uma única vez no código antes do jogo inicializar qualquer coisa. Essa função tem como intenção alocar as variaveis e coisas que não serão executadas a todo frame.

\subsection{void Update()}
Chamada a todo frame e tem como intenção atualizar as variaveis do jogo.
 
\subsection{void Draw()}
Chamada todo frame com a intenção de desenhar coisas na tela. Essa função é otimizada pra utilizar a biblioteca Graphics.

\section{Classe Transform}
\label{doc:transform}

A classe transform é um container de translação, rotação e escala, tendo também algumas funções auxilares muito usadas com essa informação.

\subsection{Transform()}
O construtor da Classe transform pode receber um vetor de translação, um quaternião de rotação e um vetor de escala.

\subsection{glm::vec3 forward()}
Esse método retorna o vetor (0, 0, -1) rotacionado da rotação da Transform.

\subsection{glm::vec3 up()}
Esse método retorna o vetor (0, 1, 0) rotacionado da rotação da Transform.

\subsection{glm::vec3 right()}
Esse método retorna o vetor (1, 0, 0) rotacionado da rotação da Transform.

\subsection{void Translate(glm::vec3 translation)}
Esse método muda a posição da Transform, fazendo uma translação definida pelo parametro \textit{translation}.

\subsection{void Rotate(float angle, glm::vec3 axis)}
Esse rotaciona a Transform, fazendo uma rotação definida pelo eixo \textit{axis} e pelo angulo \textit{angle}.

\subsection{glm::vec3 position}
A posição da Transform
\subsection{glm::vec3 rotation}
A rotação da Transform
\subsection{glm::vec3 scale}
A escala da Transform

\section{Classe Camera}
\label{doc:camera}

A camera é um objeto utilizado pela biblioteca Graphics para determinar como a cena será renderizada, toda cena deverá possuir pelo menos uma camera.

\subsection{Camera(const float fov, const float aspect\_ratio, const float near, const float far)}
O construtor da camera recebe basicamente as informaçoes sobre o \textit{frustrum}. Cria uma camera na posição (0, 0, 0) e sem rotação (olhando para o eixo z negativo).

\subsection{void Translate(glm::vec3 translation)}
Esse método muda a posição da Transform da camera, fazendo uma translação definida pelo parametro \textit{translation}.
  
\subsection{void Rotate(float angle, glm::vec3 axis)}
Esse rotaciona a Transform da camera, fazendo uma rotação definida pelo eixo \textit{axis} e pelo angulo \textit{angle}.

\subsection{void SetPosition(glm::vec3 position)}
Esse método seta a posição da Transform da camera.

\subsection{void SetRotation(glm::quat rotation)}
Esse método seta a rotação da Transform da camera.

\subsection{glm::vec3 position()}
Esse método pega a posição da Transform da camera.

\subsection{glm::quat rotation()}
Esse método pega a rotação da Transform da camera.

\section{Classe EventSystem}
\label{doc:events}

\subsection{EventSystem()}
Cria um novo event system.

\subsection{void Add(Event e, Callback callback)}
Adiciona um novo callback para quando o evento e for disparado.

\subsection{void Fire(Event e)}
Dispara o evento e.

\section{Classe KeyboardInput}
\label{doc:keyboardinput}
Essa classe serve para transformar inputs do teclado do usuário em funções do jogo.

\subsection{KeyboardInput(SDL\_KeyMapping *keymapping, const int mappingsize)}
O construtor recebe uma lista de mapeamento de teclas para um nome de ação e o tamanho dessa lista.

\subsection{void BindAction(const char *name, const InputType type, const std::function<void()> callback)}
Essa função registra a função callback para ser chamada quando a ação \textit{name}  do tipo \textit{type} for chamada.

InputType pode ser INPUT\_DOWN (quando a tecla é pressionada), INPUT\_UP (quando a tecla é solta)  e INPUT\_HOLD (quando ela está sendo segurada);

\section{Classe MouseInput}
\label{doc:mouseinput}
Essa classe serve para transformar inputs do mouse do usuário em funções do jogo.

\subsection{MouseInput(SDL\_MouseKeyMapping *keymapping, const int mappingsize)}
O construtor recebe uma lista de mapeamento de botões do mouse para um nome de ação e o tamanho dessa lista.

\subsection{void BindAction(const char *name, const InputType type, const std::function<void()> callback)}
Análoga à função BindAction da KeyboardInput.

\subsection{void BindMovement(const std::function<void(const MouseMovementData *d)> callback)}
Registra uma função que recebe informações sobre a movimentação do mouse para ser chamada toda vez que o mouse se movimentar.

\section{Classe Scheduler}
\label{doc:scheduler}
Essa classe serve para fazer chamadas de funções após um tempo em segundos.

\subsection{Scheduler()}
Cria um novo scheduler.

\subsection{void Update(SchedulerTime delta\_time)}
Atualiza o scheduler do tempo que passou. Geralmente o uso disso é chamar ela no Update do Game com delta\_time = Time::GetDelta().

\subsection{void Add(SchedulerTime dt, SchedulerTask task)}
Adiciona a função task para ser chamada após dt segundos.

\subsection{void Repeat(SchedulerTime delay, SchedulerTime dt, SchedulerTask task)}
Adiciona a função task para ser repetida a cada dt segundos, depois de um delay.

\subsection{void Repeat(SchedulerTime dt, SchedulerTask task)}
Adiciona a função task para ser repetida a cada dt segundos (com delay inicial de dt segundos).


\section{Namespace Time}
\label{doc:time}
Namespace com funções globais relacionadas a tempo.

\subsection{float GetDelta()}
Retorna o tempo que passou entre o ultimo frame e o frame atual.

\subsection{float GetTotal()}
Retorna o tempo total de jogo desde sua inicialização.

\section{Namespace Graphics}
\label{doc:graphics}

\subsection{void Cube()}
Existem duas versoes, a Cube(glm::vec3 position, glm::quat rotation = glm::quat(), glm::vec3 scale = glm::vec3(1.0f)), e a Cube(Transform transform), a primeira recebe uma posição, uma rotação e uma escala, e posiciona um cubo, a outra recebe uma Transform e posiciona um cubo baseado nas mesmas informações. Essa função é uma prova de conceito da futura função que importaria um objeto genérico, que nao foi implementada por motivos de simplicidade.

\subsection{void SetMaterial(glm::vec3 diffuse_color, glm::vec3 specular_color)}
Essa função seta o materia que vai ser usado a partir desse momento, dependendo do shader atual, o material pode ou não afetar em propriedades como como a luz interage com o objeto.

\subsection{void PointLight(glm::vec3 position, glm::vec3 color, float intensity)}
Cria uma luz pontual na posicao definida, com cor e intensidade dada pelos parametros. Envia essas informações que podem ou não ser utilizadas pelo atual shader.

\subsection{void SetAmbientLight(glm::vec3 color)}
Seta a luz ambiente para uma determinada cor. Envia essas informações que podem ou não ser utilizadas pelo atual shader.

\subsection{void SetClearColor(glm::vec3 color)}
Seta a cor que é o plano de fundo do jogo (padrão preto).

\subsection{void SetCamera(const Camera *camera)}
Faz com que a atual camera que está renderizando a cena seja a setada por essa função.

\subsection{void Clear()}
Limpa todos os poligonos criados até então.

